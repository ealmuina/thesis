\thispagestyle{empty}

\begin{center}
    \Large
    \textbf{Clasificación de señales bioacústicas empleando técnicas de aprendizaje no supervisado}

    \vspace{0.4cm}
    \large
    Eddy Nelson Almuiña Hernández

    \vspace{0.9cm}
    \textbf{Resumen}
\end{center}

La clasificación de señales bioacústicas constituye una tarea engorrosa para los especialistas debido al gran volumen de datos que deben procesar.
La aplicación de técnicas de Inteligencia Artificial ha resultado muy útil en ese sentido, sin embargo la mayor parte de los avances han sido resultado de la aplicación de algoritmos de aprendizaje supervisado, quedando un amplio rango de técnicas con poco estudio de su posible utilidad en esta área.
En este trabajo se presenta una herramienta para el análisis y clasificación de señales bioacústicas empleando técnicas de aprendizaje no supervisado.
Se comparan los resultados para diferentes algoritmos y representaciones computacionales de las señales de sonido.
Los resultados obtenidos fueron notables e indican que los algoritmos aquí mostrados pueden tener una contribución significativa en este campo.

\begin{center}
    \vspace{0.9cm}
    \textbf{Abstract}
\end{center}

The classification of bioacoustic signals is a cumbersome task for specialists due to the large volume of data they must process.
Application of Artificial Intelligence techniques has been very useful in that sense, however most of the advances have been the result of the use of supervised learning algorithms;
the possible usefulness of a broad range of techniques in this area is still barely studied.
In this work, a tool for the analysis and classification of bioacoustic signals using unsupervised learning techniques is presented.
Results are compared for different algorithms and computational representations of sound signals.
The results obtained were remarkable, and indicate that the algorithms shown here may have a significant contribution in this field.

\newpage