Se denominan algoritmos de \textit{clustering} a los algoritmos de aprendizaje no supervisado que particionan un conjunto de datos en conjuntos cuyos miembros son tan similares como sea posible.

La determinación del número de conjuntos es a menudo un problema en sí;
algunos algoritmos lo hayan como parte de su funcionamiento, mientras que otros requieren dicho valor como entrada.
Al trabajo de elegir el modelo de <<complejidad>> adecuada se le conoce como \textit{selección del modelo} y será un asunto que discutiremos más adelante. %TODO check if accomplished or change

En correspondencia con el marco teórico que emplean, una posible clasificación para este tipo de algoritmos es en:
\begin{itemize}
    \item \textbf{Probabilísticos y generativos}
    \item \textbf{Basados en distancia}
    \item \textbf{Basados en densidad}
\end{itemize}
