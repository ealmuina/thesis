\begin{table}[h]
    \centering
    \begin{tabular}{lll}
        \hline
        Especie & Segmentos & Tipos de vocalizaciones   \\ \hline
        A. jamaicensis & 10 & 1                         \\
        B. nana & 10 & 1                                \\
        E. fuscus & 10 & 1                              \\
        E. glaucinus & 10 & 1                           \\
        E. poeyi & 10 & 1                               \\
        E. sezekorni & 10 & 1                           \\
        M. blainvillei & 10 & 1                         \\
        M. minutus & 10 & 1                             \\
        M. molosus & 10 & 1                             \\
        M. redmani & 10 & 1                             \\
        M. waterhousii & 10 & 1                         \\
        N. cubanus & 10 & 1                             \\
        N. lepidus & 10 & 1                             \\
        N. leporinus & 10 & 1                           \\
        N. macrotis & 10 & 1                            \\
        N. primus & 10 & 1                              \\
        P. parnelli & 10 & 1                            \\
        P. quadridens & 10 & 1                          \\
        T. brasiliensis & 10 & 1
    \end{tabular}
    \caption{Composición del conjunto de datos de especies de murciélagos.}
    \label{table:bats-dataset}
\end{table}

\begin{table}[h]
    \centering
    \begin{tabular}{lll}
        \hline
        Especie & Segmentos & Tipos de vocalizaciones   \\ \hline
        A. cannabina & 10 & 1                           \\
        A. trivialis & 10 & 1                           \\
        C. mexicanus & 10 & 1                           \\
        C. crex & 10 & 1                                \\
        D. major & 20 & 2                               \\
        F. coelebs & 10 & 1                             \\
        J. torquilla & 20 & 2                           \\
        L. fluviatilis & 10 & 1                         \\
        L. naevia & 10 & 1                              \\
        L. megarhynchos & 10 & 1                        \\
        P. caeruleus & 10 & 1                           \\
        P. sibilatrix & 40 & 4                          \\
        P. minor & 10 & 1                               \\
        S. europaea & 20 & 2                            \\
        S. melanocephala & 10 & 1                       \\
        T. ruficollis & 20 & 2
    \end{tabular}
    \caption{Composición del conjunto de datos de especies de aves.}
    \label{table:birds-dataset}
\end{table}

\begin{table}[h]
    \centering
    \begin{tabular}{lll}
        \hline
        Especie & Segmentos & Tipos de vocalizaciones   \\ \hline
        A. domesticus & 14 & 1                          \\
        E. burdigalensis & 20 & 1                       \\
        G. gryllotalpa & 20 & 1                         \\
        G. bimaculatus & 20 & 1                         \\
        G. campestris & 17 & 1                          \\
        M. bicolor & 20 & 1                             \\
        M. roeselii & 17 & 1                            \\
        M. maculatus & 20 & 1                           \\
        T. caudata & 20 & 1                             \\
        T. viridissima & 16 & 1
    \end{tabular}
    \caption{Composición del conjunto de datos de especies de insectos.}
    \label{table:insects-dataset}
\end{table}