Para procesar una señal esta a menudo es dividida en pequeños segmentos, conocidos como \textit{tramas}\footnote{\textit{frames} en inglés.}, de longitud constante y espaciados en intervalos de tiempo iguales.
Denotamos por $N$ la cantidad de muestras de la señal que contiene una trama;
de esta forma la duración de una trama será de $N/F_s$.
Asimismo, denotamos por $M$ la cantidad de muestras en que difieren dos tramas consecutivas, conocida como \textit{tamaño de paso}, y que usualmente es menor que $N$.
A partir de estos valores podemos igualmente calcular la cantidad de muestras que dos tramas consecutivas tienen en común, como $N-M$;
y el número de tramas por segundo (\textit{frame rate}) como $F_s/M$.

Cada trama de $N$ muestras es usualmente obtenida mediante la aplicación de una función \textit{ventana} $w(n)$ a la señal, que es distinta de cero solo si $0\leq n\leq N-1$.
Dada la señal $s[n]$, una trama que comienza en la muestra $m$ es obtenida como

\begin{equation}
    \label{eq:windowing}
    s[n]_m = \begin{cases}
                 s[n + m]w(n) & 0\leq n\leq N-1 \\
                 0 & eoc.
    \end{cases}
\end{equation}

Dos de las variantes más conocidas para la función $w(n)$ son las siguientes:

\begin{itemize}
    \item \textbf{Rectangular}:
    \begin{equation*}
        w(n) = \begin{cases}
                   1 & 0\leq n\leq N-1 \\
                   0 & eoc.
        \end{cases}
    \end{equation*}
    \item \textbf{Hamming}:
    \begin{equation*}
        w(n) = 0.53836 - 0.46164 \cos\left(\frac{2\pi n}{N-1}\right)
    \end{equation*}
\end{itemize}

La elección de la ventana $w(n)$ tiene un efecto sobre los resultados de operaciones posteriores sobre las tramas obtenidas.
En la práctica, algunos métodos de procesamiento, como la transformada de Fourier producen mejores resultados cuando se emplean funciones que se aproximan a cero en sus extremos, como la de Hamming.