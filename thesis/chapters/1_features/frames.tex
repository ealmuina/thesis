Para procesar una señal esta a menudo es dividida en pequeños segmentos, conocidos como \textit{tramas}\footnote{\textit{frames} en inglés.}, de longitud constante y espaciados en intervalos de tiempo iguales.
Denotamos por $N$ la cantidad de muestras de la señal que contiene una trama;
de esta forma la duración de una trama será de $N/F_s$.
Asimismo, denotamos por $M$ la cantidad de muestras en que difieren dos tramas consecutivas, conocida como \textit{tamaño de paso}, y que usualmente es menor que $N$.
A partir de estos valores podemos igualmente calcular la cantidad de muestras que dos tramas consecutivas tienen en común, como $N-M$;
y el número de tramas por segundo (\textit{frame rate}) como $F_s/M$.

Cada trama de $N$ muestras es usualmente obtenida mediante la aplicación de una función \textit{ventana} $w(n)$ a la señal, que es distinta de cero solo si $0\leq n\leq N-1$.
Dada la señal $s[n]$, una trama que comienza en la muestra $m$ es obtenida como

\begin{equation}
    \label{eq:windowing}
    s[n]_m = \begin{cases}
                 s[n + m]w(n) & 0\leq n\leq N-1 \\
                 0 & \text{eoc.}
    \end{cases}
\end{equation}

La elección de la ventana $w(n)$ tiene un efecto sobre los resultados de operaciones posteriores sobre las tramas obtenidas.
En la práctica, algunos métodos de procesamiento, como la transformada de Fourier producen mejores resultados cuando se emplean funciones que se aproximan a cero en sus extremos, como la de Hamming.

\begin{table}[H]
    \centering
    \begin{tabular}{ll}
        \hline
        Nombre & Fórmula                                                                                                               \\ \hline
        \textit{Rectangular} & $w(n) = 1$                                                                                                            \\
        \textit{Hanning} & $w(n) = 0.50 - 0.50 \cdot \cos \left( \frac{2\pi n}{N-1} \right)$                                                     \\
        \textit{Hamming} & $w(n) = 0.54 - 0.46 \cdot \cos \left( \frac{2\pi n}{N-1} \right)$                                                     \\
        \textit{Blackman} & $w(n) = 0.42 - 0.50 \cdot \cos \left( \frac{2\pi n}{N-1} \right) + 0.08 \cdot \cos \left( \frac{4\pi n}{N-1} \right)$
    \end{tabular}
    \caption{Variantes de ventana más comunes, definidas para $n\in[0, N-1]$.}
    \label{table:window-function}
\end{table}

Una vez computados, los valores asociados a una característica de la señal para cada trama suelen ser <<generalizados>> para representar su comportamiento global.
El procedimiento varía desde la generación de nuevas características a partir de las ya calculadas~\cite{Giret11}, hasta la simplificación de estas a la media y varianza de cada una de sus componentes.
Esto es especialmente importante para el empleo de dichos vectores en algoritmos de inteligencia artificial, puesto que las señales no siempre se descomponen en un mismo número de tramas;
mientras que aplicando este método, el tamaño de los vectores de características sí será igual para todas, lo cual es un requisito de muchos de dichos algoritmos.