El primer paso para el análisis de una señal suele ser su descomposición en frecuencias.
Una trama $x[n]$, de longitud $N$, tiene una representación en dicho dominio que puede ser obtenida mediante la aplicación de la transformada discreta de Fourier (DFT por sus siglas en inglés):

\begin{equation}
    \label{eq:DFT}
    X[f] = \sum_{n=0}^{N-1}{x[n]e^{\frac{-i2\pi fn}{N}}}
\end{equation}

El espectro $X[f]$ puede ser transformado de vuelta al dominio del tiempo aplicando la transformada discreta inversa de Fourier (IDFT):

\begin{equation}
    \label{eq:IDFT}
    x[n] = \frac{1}{N}\sum_{f=0}^{N-1}{X[f]e^{\frac{i2\pi fn}{N}}}
\end{equation}

Al ser $x[n]$ un vector de valores reales, podemos aplicar sobre este una propiedad de la DFT que plantea que el vector $X[f]$ es, por tanto, conjugado simétrico, por lo que suelen considerarse solamente los últimos $\lceil (N+1)/2 \rceil$ valores de este.

Podemos definir $X[f,t]$ como la matriz compuesta por las DFTs correspondientes a cada una de las tramas de una señal, donde el valor en la posición $(f, t)$ corresponde a la amplitud de la frecuencia $f$ en la trama $t$.
A partir de esta matriz se construye el espectrograma de la señal.
