Este tipo de características son de las más simples y se computan directamente a partir de la señal digitalizada.
Algunas de las más comunes son:

\begin{enumerate}
    \item \textbf{Short time energy}: Da una medida de la cantidad de energía presente en la señal.
    \[
        STE = \frac{1}{N}\sum_{n=0}^{N-1}{x[n]^2}
    \]
    \item \textbf{Zero crossing rate}: Está dado por el número de veces que la amplitud de la señal cambia de signo.
    \[
        ZCR = \frac{1}{2}\sum_{n=1}^{N-1}{|\text{sign}(x[n]) - \text{sign}(x[n-1])|}
    \]
    donde $\text{sign}(x[n])$ es el signo de la amplitud de la muestra $n$-ésima de la señal.
    \item \textbf{Temporal centroid}: Centro de gravedad del oscilograma.
    \[
        TC = \frac{\sum_{n=0}^{N-1}{n*x[n]}}{\sum_{n=0}^{N-1}{x[n]}}
    \]
    \item \textbf{Temporal width}: Dispersión de las intensidades en torno al centroide.
    \[
        TW = \sqrt{\frac{\sum_{n=0}^{N-1}{(n-TC)^2 x[n]^2}}{\sum_{n=0}^{N-1}{x[n]^2}}}
    \]
    \item \textbf{Temporal flatness}: Da una medida de la variación de la intensidad del sonido en el transcurso del tiempo.
    \[
        TF = \frac{(\prod_{n=0}^{N-1}{|x[n]})^{\frac{1}{N}}}{\frac{1}{N}\sum_{n=0}^{N-1}{x[n]}}
    \]
    Los valores próximos a 1 indican que la intensidad del sonido permanece con poca variación, mientras que lo contrario corresponde a valores cercanos a 0.
    \item \textbf{Autocorrelación}: Coeficientes que pueden ser interpretados como la distribución espectral de la señal en el dominio del tiempo.
    Los primeros $k$ coeficientes pueden ser obtenidos mediante la fórmula
    \[
        R(k) = \frac{\sum_{n=0}^{N-k-1}{x[n]x[n+k]}}{\sqrt{\sum_{n=0}^{N-k-1}{x[n]^2}}\sqrt{\sum_{n=0}^{N-k-1}{x[n+k]^2}}}
    \]
    \item \textbf{Duración}: Longitud del segmento en el tiempo.
    \[
        L = t_{\max} - t_{\min}
    \]
\end{enumerate}
