Las características incluidas en este grupo describen el grado de armonicidad de la señal, es decir, su composición en términos de señales puramente armónicas.

\subsection{Fundamental Frequency}\label{subsec:fundamentalFrequency}

Este descriptor provee una estimación de la \textit{frecuencia fundamental} (FF) de la señal en cada una de sus tramas.
Es la frecuencia más baja del espectro de frecuencias tal que las frecuencias dominantes pueden expresarse como múltiplos de esta.
Puede, por tanto, considerarse la FF como la frecuencia de la señal armónica que mejor representa a la señal en cuestión.

Existen numerosas variantes para la estimación de la FF de una señal~\cite{Kim05}, siendo la propuesta en~\cite{Cheveigne02} una de las más populares.
Esta técnica, también conocida como \textit{algoritmo YIN};
se basa, a grandes rasgos, en encontrar el valor del período $\tau$ que minimiza la siguiente función para cada trama $t$~\cite{Gerhard03-2}:

\begin{gather}
    \label{eq:YIN}
    d_t'(\tau) = \begin{cases}
                     1 & \tau = 0 \\
                     \frac{d_t(\tau)}{\frac{1}{\tau}{\sum_{j=1}^{\tau}{d_t(j)}}} & eoc.
    \end{cases}\\
    d_t(\tau) = \sum_{i=0}^{N-1}{(x[i]-x[i+\tau])^2}
\end{gather}

Luego, la frecuencia fundamental estará dada por la expresión $FF = 1/\tau$.

\subsection{Inharmonicity}\label{subsec:inharmonicity}

La \textit{inharmonicity} (INH) representa la divergencia de las frecuencias que componen la señal respecto a una señal puramente armónica.

%TODO describe computation of inharmonicity

Los valores de la INH varían entre 0 (señal puramente armónica) y 1 (señal no armónica).

\subsection{Odd to Even Harmonic Energy Ratio}\label{subsec:oddToEvenHarmonicEnergyRatio}

(OEHR)

\subsection{Tristimulus}\label{subsec:tristimulus}

(TR)

\subsection{Harmonic Spectral Shape}\label{subsec:harmonicSpectralShape}

% TODO Harmonic peaks

\subsection{Harmonic Spectral Centroid}\label{subsec:harmonicSpectralCentroid}

(HSC)

\subsection{Harmonic Spectral Spread}\label{subsec:harmonicSpectralSpread}

(HSS)

\subsection{Harmonic Spectral Roll-off}\label{subsec:harmonicSpectralRoll-off}

(HSRO)


