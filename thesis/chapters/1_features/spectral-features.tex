Estas características se derivan a partir de la representación de la señal en el dominio de las frecuencias.
Algunas de las más comunes son:

\begin{enumerate}
    \item \textbf{Energía}: Se computa para una frecuencia específica como la suma de los cuadrados de las magnitudes que esta tiene en el tiempo.
    Es, en otras palabras, la suma de los cuadrados del módulo de los valores por cada fila de la matriz $X[f,t]$.
    \item \textbf{Spectral centroid}: Constituye el centroide del espectro de frecuencias.
    \[
        SC = \frac{\sum_{k=0}^{K-1}{f(k)|X[k]|}}{\sum_{k=0}^{K-1}{|X[k]|}}
    \]
    donde $K = \lceil (N+1)/2 \rceil$ es la longitud del vector obtenido aplicando la DFT a la trama, y $f(k)$ frecuencia asociada a la posición $k$-ésima de este.
    \item \textbf{Spectral width}: Medida de la dispersión del espectro de frecuencias en torno a su centroide (SC).
    \[
        SW = \sqrt{\frac{\sum_{k=0}^{K-1}{(f(k)-SC)^2 |X[k]|^2}}{\sum_{k=0}^{K-1}{|X[k]|^2}}}
    \]
    \item \textbf{Spectral flatness}: Permite discriminar si hay eventos harmónicos presentes en la señal o esta se encuentra compuesta solamente por ruido.
    \[
        SF = \frac{(\prod_{k=0}^{K-1}{|X[k]|})^{\frac{1}{K}}}{\frac{1}{K}\sum_{k=0}^{K-1}{|X[k]|}}
    \]
    Los valores de $SF$ próximos a 1 indican que la señal se compone de ruido, mientras que una señal harmónica corresponde a un valor cercano a 0.
    \item \textbf{Spectral roll-off}: Se define como la frecuencia bajo la cual se encuentra un porcentaje específico (usualmente entre el 85\% y el 99\%) de la energía espectral total.
    \item \textbf{Spectral flux}: Caracteriza la variación dinámica de la información espectral.
    Se calcula como la norma de la diferencia entre los espectros de dos tramas consecutivas.
    \item \textbf{Rango de frecuencias}: Máxima ($f_{\max}$) y mínima ($f_{\min}$) frecuencias presentes en la señal.
\end{enumerate}