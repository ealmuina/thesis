Contrario a las medidas supervisadas, las de validación no supervisada no requieren el conocimiento previo de las categorías reales a las que pertenecen los objetos del conjunto de datos.
Esto las hace muy convenientes, especialmente considerando que en muchas ocasiones el uso de los algoritmos de clustering ocurre precisamente en escenarios en los que no se dispone de dichas categorías.
Y es gracias a este tipo de medidas que los resultados pueden ser evaluados en tales circunstancias.

\section{Cohesión y Separación}\label{sec:cohesiónYSeparación}

En general, para evaluar un conjunto de $K$ clusters suele emplearse la suma ponderada de las evaluaciones de los clusters individuales~\cite{Tan05}:

\begin{equation}
    \label{eq:overall-validity}
    overall\ validity = \sum_{i=1}^{K}{w_i\ validity(C_i)}
\end{equation}

La evaluación del cluster $C_i$ está dada por la función $validity(C_i)$, cuyo peso es indicado por el coeficiente $w_i$.
Existen numerosas variantes para seleccionar dicha función de evaluación y los pesos, algunas de las cuales aparecen reflejadas en la tabla~\ref{table:validity-weights}.

Dos funciones a partir de las que puede construirse $validity(C_i)$ son \textbf{cohesión} y \textbf{separación}, que respectivamente miden lo próximos que están entre sí los puntos de $C_i$ y lo lejos que se encuentran de puntos fuera del cluster.
Pueden ser planteadas siguiendo dos criterios:

\begin{itemize}
    \item Entre pares de puntos:
    \begin{align}
        cohesion_1(C_i) & = \sum_{\substack{x\in C_i \\ y\in C_i}}{proximity(x,y)} \\
        separation_1(C_i, C_j) & = \sum_{\substack{x\in C_i \\ y\in C_j}}{proximity(x,y)}
    \end{align}

    \item Entre cada punto y un punto <<distinguido>> $c_i$ como, por ejemplo, el centroide o la media de los puntos del cluster $C_i$:
    \begin{align}
        \label{eq:selected-point-cohesion}
        cohesion_2(C_i) & = \sum_{x\in C_i}{proximity(x,c_i)} \\
        \label{eq:selected-point-separation}
        separation_2(C_i, C_j) & = proximity(c_i,c_j)
    \end{align}
\end{itemize}

La función $proximity$ puede indicar la distancia, similaridad, o cualquier otro criterio que se desee emplear en correspondencia con el problema específico en que se aplique.

\begin{table}[H]
    \centering
    \begin{tabular}{rl}
        \hline
        Función de validación ($validity(C_i)$)                        & Peso ($w_i$)                            \\                 \hline
        $cohesion_1(C_i)$                                              & $1/n_i$                                 \\
        $cohesion_2(C_i)$                                              & 1                                       \\
        $proximity(c_i , c)$                                           & $n_i$                                   \\
        $\sum_{\substack{j=1 \\ j\neq i}}^{K}{separation_1(C_i, C_j)}$ & $\frac{1}{cohesion_1(C_i)}$
    \end{tabular}
    \caption{Algunas de funciones de validación y pesos que pueden emplearse en la evaluación no supervisada de los resultados de algoritmos de clustering. $c$ centroide (o punto distinguido) del conjunto de datos completo.
    (Adaptada de~\cite{Tan05})}
    \label{table:validity-weights}
\end{table}

\section{Coeficiente de silueta}\label{sec:coeficienteDeSilueta}

El coeficiente de silueta~\cite{Rousseeuw87, Tan05} combina la cohesión y la separación en una medida común.
Su cálculo se realiza independientemente para cada punto $x$ aplicando el siguiente algoritmo:

\begin{algorithm}
    \caption{Coeficiente de silueta}
    \label{algorithm:silohuette-coefficient}
    Calcular $a_x$, distancia promedio del punto a todos los demás puntos de su cluster\;
    Calcular, para cada cluster al que no pertenece el punto, la distancia promedio de este último a sus elementos, y conservar el menor de dichos promedios, $b_x$\;
    El coeficiente de silueta de $x$ estará dado entonces por la expresión:
    \begin{equation*}
        s_x = \frac{b_x - a_x}{\max{(a_x, b_x)}}
    \end{equation*}
\end{algorithm}

El valor de este medidor varía entre -1 y 1.
Los valores negativos corresponden al caso en que $a_x$ es mayor que $b_x$, o sea que la distancia a los puntos de su cluster es mayor que la que lo separa del cluster más cercano.
En cambio, cuando el valor es positivo, se tiene que el punto está mucho más próximo a puntos de su mismo cluster que a los de clusters diferentes.

Una vez conocidos los coeficientes de silueta de cada punto, su promedio puede ser utilizado para medir la calidad general del conjunto de clusters.

