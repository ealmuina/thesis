Las medidas de validación supervisadas o \textbf{externas}, parten de la premisa de que se disponen las categorías reales de los elementos del conjunto de datos sobre el que se aplicó el algoritmo de clustering.
Haciendo uso de este conocimiento, pueden construirse clusters ideales, estableciendo una correspondencia biyectiva entre clusters y categorías.
La tarea de los criterios de validación supervisados es comparar el resultado de la aplicación de un algoritmo de clustering con dicho conjunto de clusters ideales.

De acuerdo al enfoque que adopta cada criterio para establecer la comparación, estos pueden agruparse en dos conjuntos:

\begin{itemize}
    \item \textbf{Orientados a la clasificación}: Son aquellas medidas que evalúan la composición de los clusters respecto a qué tan cerca se encuentran de contener elementos pertenecientes a una única clase.
    \item \textbf{Orientados a la similaridad}: Estos criterios valoran cuánto se cumple que dos objetos de una misma categoría se encuentren en el mismo cluster y viceversa.
\end{itemize}

\subsection{Medidas orientadas a la clasificación}\label{subsec:medidasOrientadasALaClasificación}

Este grupo de medidas constituye una adaptación de criterios empleados tradicionalmente para la evaluación de algoritmos de aprendizaje supervisado.
A continuación se mencionan algunas de ellas.

\subsubsection{Entropía}

La \textit{entropía} de un conjunto es el grado en que este está constituido por elementos de una única clase.
La entropía de un conjunto de datos $X$ cuyos elementos se encuentran categorizados por el conjunto de clases $L$ puede calcularse mediante la fórmula:

\begin{equation}
    \label{eq:entropy}
    H(X, L) = -\sum_{l=1}^{|L|}{\frac{|X_l|}{|X|}\log{\left( \frac{|X_l|}{|X|} \right)}}
\end{equation}

\noindent
donde $|L|$ es el total de categorías en que están agrupados los elementos del conjunto, $|X_l|$ la cantidad de elementos que pertenecen a la categoría $l$, y $|X|$ el total de elementos del conjunto.

De igual manera, la entropía de un conjunto de clusters conocidas las categorías reales a las que pertenecen sus elementos, se puede calcular aplicando la siguiente expresión:

\begin{equation}
    \label{eq:clustering-entropy-conditional}
    H(X, K|C) = -\sum_{i=1}^{|K|}\sum_{j=1}^{|C|}{\frac{n_{ij}}{n}\log \left( \frac{n_{ij}}{n_i} \right)}
\end{equation}

\noindent
donde $K$ es el conjunto de clusters, $C$ el conjunto de categorías, $n$ total de elementos del conjunto de datos $X$, $n_i$ la cantidad de elementos del cluster $i$, y $n_{ij}$ el número de elementos que pertenecen simultáneamente al cluster $i$ y la categoría $j$.
Esta expresión puede simplificarse si es vista como la suma de las entropías de cada uno de los clusters, ponderada por los tamaños de estos, convirtiéndose así en:

\begin{equation}
    \label{eq:clustering-entropy}
    e = H(X, K|C) = \sum_{i=1}^{|K|}{\frac{n_i}{n}H(K_i,C)}
\end{equation}

\noindent
donde $K_i$ es el subconjunto del conjunto de datos constituido por los elementos que forman parte del cluster $i$.

%TODO Check this statement
Los valores de este medidor se encuentran en el rango $[0, \log|C|]$, siendo 0 el indicador de que todos los elementos se agruparon correctamente.

\subsubsection{Pureza, Precisión y Recobrado}

La \textbf{pureza} de un cluster mide lo cerca que este se encuentra de contener objetos de una única categoría.
Se define por las fórmulas:

\begin{gather}
    \label{eq:cluster-purity}
    p_i =\max_{j}{p_{ij}} \\
    \label{eq:purity}
    purity = \sum_{i=1}^{K}{\frac{n_i}{n}p_i}
\end{gather}

\noindent
donde  $p_{ij} = n_{ij}/n_i$ es la probabilidad de que un elemento del cluster $i$ pertenezca a la categoría $j$;
(\ref{eq:cluster-purity}) determina la pureza de un cluster, y~(\ref{eq:purity}) la de un conjunto de clusters.

La \textbf{precisión} es la fracción de los elementos de un cluster que pertenecen a la misma clase.
La precisión del cluster $i$ respecto a la clase $j$ está dada por la expresión $precision(i,j) = $.

El \textbf{recobrado} mide el grado en que un cluster contiene todos los elementos que pertenecen a una clase dada.
Para el cluster $i$ y la categoría $j$, su expresión es $recall(i,j) = n_{ij}/n_j$, donde $n_j$ es el número de elementos del conjunto de datos incluidos en la clase $j$.

Los tres criterios mencionados toman valores en el rango $(0,1]$, donde un valor más alto indica un mejor resultado.

\subsubsection{Medida F}

Haciendo uso de los conceptos de precisión y recobrado, se define la medida F como una combinación de estos, que busca medir el grado en que un cluster está constituido solamente por elementos de una única clase y a su vez contiene a todos los elementos de dicha clase.
La medida F del cluster $i$ respecto a la clase $j$ está dada por:

\begin{equation}
    \label{eq:F-measure}
    F(i,j) = \frac{2 \cdot precision(i,j) \cdot recall(i,j)}{precision(i,j) + recall(i,j)}
\end{equation}

Los valores de la medida F oscilan en el intervalo $(0,1]$, correspondiéndose los valores más altos con los mejores resultados.

\subsubsection{Homogeneidad, Completitud y Medida V}

Con propósitos semejantes a los trazados por la precisión y el recobrado, pero generalizando la información para todas las categorías, fueron diseñados los criterios de \textbf{homogeneidad} ($h$) y \textbf{completitud} ($c$) respectivamente.
Estos están dados por las siguientes fórmulas:

\begin{gather}
    h = 1 - \frac{H(X, C|K)}{H(X, C)} \\
    c = 1 - \frac{H(X, K|C)}{H(X, K)}
\end{gather}

\noindent
donde $X$ es el conjunto de datos clasificados por las categorías $C$, y $K$ los clusters obtenidos sobre dicho conjunto.
Para el cálculo de $H(C|K)$ se emplea~\ref{eq:clustering-entropy-conditional} de forma simétrica, es decir, considerando los clusters como categorías y viceversa.

A partir de la homogeneidad y la completitud se puede definir la \textbf{Medida V}, de modo equivalente al que se definió la medida F;
o sea, mediante la fórmula:

\begin{equation}
    \label{eq:V-measure}
    V = \frac{2 \cdot h \cdot c}{h + c}
\end{equation}

De forma semejante a lo que sucede con los respectivos criterios equivalentes a la homogeneidad, completitud y medida V;
estos criterios toman valores ubicados en el intervalo $[0,1]$ donde a medida que un valor es más alto, mejor ha sido evaluado el resultado.