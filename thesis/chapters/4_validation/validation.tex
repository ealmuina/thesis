La evaluación de los resultados producidos por un algoritmo es un paso usual de la aplicación de técnicas de aprendizaje supervisado a un problema.
Dado que se dispone de conocimiento de la categoría real de los elementos a clasificar la evaluación del modelo suele darse mediante la comparación entre las categorías que este asigna a un conjunto de prueba y las que realmente tenían sus elementos.

Un procedimiento semejante no puede ser usado para validar los resultados de un algoritmo de aprendizaje no supervisado, puesto que su naturaleza radica en el desconocimiento de las categorías por parte del algoritmo.
La tarea de estos algoritmos consiste en agrupar los datos siguiendo determinados criterios, siempre sin asignar una categoría conocida de antemano a cada elemento.
De ahí que comparar los resultados con las categorías a las que realmente pertenecen los datos, aun si estas son conocidas, no es aplicable siguiendo las pautas de los algoritmos supervisados.

Sin embargo, diversos criterios han sido desarrollados con el propósito de evaluar los resultados de la aplicación de un algoritmo no supervisado sobre un conjunto de datos determinado;
lo que a su vez permite establecer comparaciones entre estos con la finalidad de determinar el modelo que mejor se ajusta a cada situación particular.

En este capítulo se mencionan algunos de los criterios de validación más populares, agrupados según si requieren o no el conocimiento de la categoría real de cada observación del conjunto de datos.


