Frecuentemente sucede que un número de grande de dimensiones en la representación del conjunto de datos, lejos de aportar un beneficio en su procesamiento, dificulta la tarea y conlleva a resultados donde la presencia de <<ruido>>\footnote{La existencia de ruido en un conjunto de datos suele deberse a imperfecciones de las tecnologías empleadas para confeccionarlo, así como a la naturaleza del origen de los propios datos (por ejemplo, la grabación de sonidos en un ambiente donde varias vocalizaciones animales se solapan entre ellas y con otros sonidos de origen natural como los del viento y la lluvia).} en los datos produce errores.
El uso de conjuntos de datos de alta dimensionalidad casi siempre suma, por tanto, tiempo adicional a la ejecución de los algoritmos y errores a sus respuestas.
Por otra parte, el análisis de los datos por parte de seres humanos se dificulta cuando estos sobrepasan las dos o tres dimensiones, especialmente porque se imposibilita el uso de gráficas para visualizar su comportamiento.
De ahí que se haga muy compleja la interpretación por parte del usuario de los resultados producidos por un algoritmo (especialmente los \textit{no supervisados}).

Para dar solución a las problemáticas mencionadas, se desarrollaron las técnicas de \textit{reducción de dimensiones}, que se proponen eliminar atributos irrelevantes o con significativa presencia de ruido, así como aquellos que solo aportan información redundante.

\section{Análisis de Componentes Principales}\label{subsec:PCA}
El \textit{análisis de componentes principales} (PCA\footnote{\textit{Principal component analysis} en inglés.}) es un procedimiento estadístico ampliamente empleado para la reducción de dimensiones de un conjunto de datos.

La \textit{covarianza} mide la dependencia lineal entre dos variables aleatorias $x,y\in \mathbb{R}^N$.
Valores muy grandes o muy pequeños indican una fuerte dependencia entre las variables, respectivamente directa (a grandes valores de una corresponden grandes valores de la otra) o inversa (a grandes valores de una corresponden pequeños valores de la otra).
Puede calcularse como:

\begin{equation}
    \label{eq:covariance}
    \sigma_{x,y} = \frac{1}{N}\sum_{i=1}^{N}{(x_i - \bar{x})(y_i - \bar{y})}
\end{equation}

En un conjunto de datos, cada componente de sus elementos puede ser considerada desde el punto de vista estadístico como una variable aleatoria, y por tanto usada para el análisis de las covarianzas entre ella y el resto de componentes.
Para ello se construye la llamada \textit{matriz de covarianza}, que tiene la siguiente forma:

\begin{equation}
    \label{eq:covariance-matrix}
    \Sigma_X = \begin{bmatrix}
                   \sigma_{1,1} & \sigma_{1,2} & \ldots & \sigma_{1,n} \\
                   \sigma_{2,1} & \sigma_{2,2} & \ldots & \sigma_{2,n} \\
                   \vdots & \vdots & \ldots & \vdots \\
                   \sigma_{n,1} & \sigma_{n,2} & \ldots & \sigma_{n,n}
    \end{bmatrix}
\end{equation}

\noindent
donde $\sigma_{i,j}$ es el valor de la covarianza entre las componentes $i$ y $j$ del conjunto de datos $X$.

Los valores en las diagonales de la matriz corresponden a la varianza de cada una de las dimensiones.
El objetivo del PCA es transformar el conjunto de datos a un espacio donde las componentes sean linealmente independientes unas de otras, lo que implica que la matriz correspondiente sea una matriz diagonal.
En otras palabras, se persigue encontrar la matriz $W$ tal que $\Sigma_Y = \Sigma_X W^T$, donde $Y$ es la nueva representación conjunto de datos.



%\section{Manifold Learning}\label{sec:manifoldLearning}
%El análisis de componentes principales no produce resultados de calidad si la relación entre las variables aleatorias es no lineal.
Los algoritmos de \textit{manifold learning} convierten los conjuntos de datos en otros de menor dimensionalidad conocidos como \textbf{manifolds}, obtenidos aplicando transformaciones no lineales sobre el espacio original de estos.

A continuación se mencionan algunos de los algoritmos de manifold learning más conocidos:

\begin{itemize}
    \item \textbf{Multi-dimensional scaling} (MDS): es una técnica que tiene como objetivo obtener una representación del conjunto de datos en un espacio de menos dimensiones, preservando las distancias existentes entre los elementos en el espacio original.\\
    En otras palabras, se busca minimizar la expresión:
    \[
        \sum_{i=1}^{N}\sum_{j=i+1}^{N}{d_{ij} - \hat{d}_{ij}}
    \]
    dado un conjunto de datos de $N$ elementos, donde $d_{ij}$ y $\hat{d}_{ij}$ son las distancias entre los elementos $i$ y $j$ en los espacios original y reducido respectivamente.

    \item \textbf{Isomap}: Puede considerarse como una extensión de MDS; que a diferencia de este último, se basa en la distancia \textit{geodésica} entre los puntos.\\
    Para ello se construye un grafo donde existirá una arista entre los puntos $i$ y $j$ si estos son \textit{vecinos} en el espacio original, que tendrá un peso igual a la distancia euclidiana entre los puntos.
    Luego la distancia geodésica entre cualquier par de puntos se corresponderá con el camino de costo mínimo entre estos en el grafo.
    Y empleando esta se aplica MDS\@.

    \item \textbf{Locally Linear Embedding} (LLE):
\end{itemize}



