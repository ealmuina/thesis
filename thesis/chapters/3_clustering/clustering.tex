Se denominan algoritmos de \textit{clustering} a los algoritmos de aprendizaje no supervisado que agrupan los elementos de un conjunto de datos en conjuntos (clusters) de modo que los objetos pertenecientes a un mismo cluster sean más similares que aquellos en clusters distintos.
Su aplicación permite simplificar la estructura del conjunto de datos, convirtiéndolo en uno más pequeño y fácilmente manipulable;
o como en el caso que ocupa este trabajo, encontrar grupos de especial significación para el problema, como pueden ser agrupaciones de segmentos de audio correspondientes a individuos de una misma especie animal.

% Wikipedia
Los algoritmos de clustering difieren significativamente en su noción de qué constituye un cluster y cómo hallarlos eficientemente.
Algunas de las nociones de cluster más populares incluyen grupos con pequeñas distancias entre sus integrantes, áreas de alta densidad en el espacio de datos, o distribuciones estadísticas particulares.
El algoritmo de clustering más apropiado para un problema, así como su configuración de parámetros (incluyendo la función de distancia a emplear, el umbral de densidad o el número de clusters esperado) es altamente dependiente de las características del conjunto de datos y del uso que se desea dar a los resultados obtenidos.

La determinación del número de conjuntos es a menudo un problema en sí;
algunos algoritmos lo hayan como parte de su funcionamiento, mientras que otros requieren dicho valor como entrada.
A la labor de selección del modelo de <<complejidad>> adecuada se le conoce como \textit{selección del modelo} y las diferentes medidas para la evaluación de los resultados producidos por un modelo dado es un tema que abordamos más adelante en este trabajo.

En las siguientes secciones explicaremos los principales algoritmos de clustering que serán empleados en este trabajo.

\section{Clustering Particional}\label{sec:clusteringParticional}
A continuación es discutido el algoritmo de clustering conocido como K-Means, uno de los más simples y eficientes existentes en la literatura.
Luego de describir en detalle el algoritmo, se analizan algunos de los principales factores que influyen sobre sus resultados.
Finalmente, se presenta una variación de K-Means que busca disminuir la complejidad computacional del algoritmo.

\subsection{K-Means}\label{subsec:k-means}

K-Means~\cite{MacQueen67} es el algoritmo de clustering particional más empleado~\cite{Aggarawal13}.
Comienza seleccionando $K$ puntos representativos como \textit{centroides} iniciales, donde $K$ es un parámetro manualmente especificado por el usuario, siendo este el número deseado de clusters a obtener.
Cada punto del conjunto de datos es luego asignado al centroide más cercano basándose en una medida de proximidad determinada.
Una vez se han formado los clusters, los centroides para cada cluster son actualizados a un nuevo punto.
De manera iterativa, el algoritmo repite estos dos pasos hasta que los centroides no cambien o algún criterio de convergencia alternativo sea cumplido.
K-Means es un algoritmo \textit{greedy} con convergencia garantizada a un mínimo local~\cite{Selim84} pero, visto como un problema de optimización, ha sido demostrado que hallar el mínimo de su función objetivo es NP-Hard~\cite{Manning08}.
En la práctica, suele usarse como criterio de convergencia una versión relajada, continuándose las iteraciones hasta que menos del 1\% de los puntos cambien de cluster.

\begin{algorithm}
    \caption{K-Means}
    \label{algorithm:KMeans}
    Seleccionar $K$ puntos como centroides\;
    \Repeat{Se cumple criterio de convergencia}{
    Formar $K$ clusters asignando cada punto al centroide más próximo\;
    Recomputar el centroide de cada cluster\;
    }
\end{algorithm}

La elección de la medida de proximidad para calcular el centroide más próximo a cada punto puede afectar significativamente las asignaciones y la calidad de la solución final.
Medidas como la distancia Manhattan (norma $L_1$), la distancia euclidiana (norma $L_2$) y la similitud coseno son frecuentemente empleadas, especialmente la segunda.
Tanto la medida de proximidad como el valor de $K$ son determinantes en la configuración de clusters producida por K-Means.

Si se analiza este algoritmo como un problema de optimización, entonces estaría minimizándose la función objetivo de K-Means conocida como Suma de Errores Cuadráticos (SSE por sus siglas en inglés), cuya formulación matemática se presenta a continuación.

Dado un conjunto de datos $D={x_1,x_2,\dots,x_N}$ de $N$ puntos, y denotado el conjunto de clusters obtenido tras aplicar K-Means como $C={C_1,C_2,\dots,C_k,\dots,C_K}$;
la SSE para $C$ es definida en la ecuación~(\ref{eq:SSE}) donde $c_k$ es el centroide del cluster $C_k$.

\begin{equation}
    \label{eq:SSE}
    SSE(C)=\sum_{k=1}^{K}{\sum_{x_{i}\in C_k}{dist(x_i, c_k)^2}}
\end{equation}

En otras palabras, se calcula el error de cada punto de los datos, es decir, su distancia al centroide más próximo, y luego es computada la suma de los cuadrados de dichos errores.
Dados dos conjuntos de clusters obtenidos aplicando dos diferentes corridas de K-Means, sería preferible conservar el de menor SSE puesto que esto significaría que los centroides hallados en esa corrida constituyen una mejor representación de los puntos en sus clusters.
De ahí que el resultado de minimizar la función SSE represente el conjunto de clusters óptimo.

Los centroides que minimizan la SSE son la media de los puntos de cada cluster~\cite{Tan05}.
El centroide del $k$-ésimo cluster, $c_k$, quedaría entonces definido según la ecuación~(\ref{eq:centroid}).

\begin{equation}
    \label{eq:centroid}
    c_{k}=\frac{1}{|C_k|}\sum_{x_{i}\in C_k}{x_i}
\end{equation}

Los pasos 3 y 4 del algoritmo~\ref{algorithm:KMeans} directamente intentan minimizar la SSE. El paso 3 forma clusters asignando los puntos al centroide más cercano, lo que minimiza el SSE de dicho conjunto de centroides.
Asimismo, el paso 4 recomputa los centroides, produciendo un nuevo conjunto de menor SSE, en correspondencia con la ecuación ~(\ref{eq:centroid}).
Sin embargo, como se mencionó anteriormente, estos pasos solamente garantizan la convergencia de K-Means a un mínimo local de la función SSE, puesto que la optimizan partiendo de una selección específica de centroides y cantidad de estos, en lugar de todas las posibles opciones.

\subsubsection{Selección de centroides iniciales}

En~\cite{MacQueen67} se propone un simple método de inicialización consistente en seleccionar los $K$ centroides de modo aleatorio.
Este es ampliamente usado en la literatura por su sencillez, aunque tiene la desventaja de que puede producir resultados muy diferentes en varias corridas del algoritmo, algunos de mayor calidad que otros.

Se han popularizado otras variantes de selección de los centroides, con el propósito de aumentar la efectividad y consistencia en los resultados de K-Means.
Uno de ellos es tomar una muestra de puntos y agruparlos empleando una técnica de clustering jerárquico.
Una vez formados $K$ clusters, se toman sus centroides y se inicializa K-Means con estos.
Este enfoque a menudo ofrece buenos resultados, pero solamente resulta práctico si la muestra tomada es relativamente pequeña (de un orden entre $10^2$ y $10^3$) y $K$ es relativamente pequeño comparado con el tamaño de dicha muestra~\cite{Tan05}.

Otra variante, conocida como \textbf{K-Means++}~\cite{Arthur07}, consiste en primeramente seleccionar un punto de manera aleatoria o tomando el centroide de todos los puntos.
Luego, se selecciona el punto más alejado de los centroides formados con anterioridad y se repite este paso hasta obtener $K$ centroides iniciales.

\subsubsection{Estimación del número de clusters}

K-Means es un algoritmo extremadamente dependiente del valor de $K$ seleccionado por el usuario.
La decisión de tal número constituye uno de los mayores desafíos, si no el mayor, al hacer uso de este algoritmo.
Es por esto que numerosos trabajos se han enfocado en el área de determinar el $K$ más apropiado, y varios métodos han sido desarrollados con tal propósito.
A continuación se mencionan algunos de los más generalizados.

\begin{enumerate}
    \item \textbf{Índice de Calinski-Harabasz}~\cite{Calinski74}: Está definido por la ecuación~(\ref{eq:CH}):
    \begin{equation}
        \label{eq:CH}
        CH(K)=\frac{\frac{B(K)}{K-1}}{\frac{W(K)}{N-K}}
    \end{equation}
    donde $N$ representa la cardinalidad del conjunto de datos.
    El número de clusters es seleccionado maximizando la función dada en la ecuación~(\ref{eq:CH}).
    $B(K)$ y $W(K)$ constituyen las sumas de los cuadrados de las distancias intra e inter-cluster respectivamente (dados $K$ clusters).

    \item \textbf{Estadística de Brecha}\footnote{\textit{Gap Statistic} en inglés.}~\cite{Tibshirani01}: En este método se generan $B$ conjuntos de datos que sigan una distribución uniforme en el mismo intervalo que el original.
    Sea $W_{b}^{*}(K)$ la suma de los cuadrados de las distancias intra-cluster del $b$-ésimo conjunto de datos, se plantea entonces la siguiente ecuación:
    \begin{equation}
        Gap(K) = \frac{1}{B} \times\sum_{b}{\log(W_{b}^{*}(K)) - \log(W(K))}
    \end{equation}
    El número de clusters seleccionado es el menor valor de $K$ que satisfaga la ecuación~(\ref{eq:Gap}):
    \begin{equation}
        \label{eq:Gap}
        Gap(K) \geq Gap(K+1) - S_{k+1}
    \end{equation}
    donde $S_{k+1}$ es el valor de la desviación estándar de $\log(W_{b}^{*}(K+1))$.

    \item \textbf{Criterio de Información de Akaike (AIC)}~\cite{Yeung01}: Sea $M$ el número de dimensiones del conjunto de datos, $K$ se calcula a partir de la ecuación~(\ref{eq:AIC}).
    \begin{equation}
        \label{eq:AIC}
        K=argmin_{K}[SSE(K)+2M K]
    \end{equation}

    \item \textbf{Coeficiente de Silueta}\footnote{\textit{Silhouette Coefficient} en inglés.}~\cite{Kaufman90}: Su formulación considera tanto la distancia intra-cluster como la inter-cluster.
    Para un punto dado $x_i$, primero se calcula el promedio de las distancias de este a todos los puntos del mismo cluster ($a_i$).
    Luego por cada cluster que no contiene a $x_i$, se computa el promedio de las distancias de $x_i$ a sus integrantes ($b_i$).
    Usando estos dos valores, se estima el coeficiente de silueta de un punto como el cociente entre su diferencia y el mayor de ambos.
    El promedio de todos los coeficientes en el conjunto de datos puede ser empleado para evaluar la calidad de un clustering.
    Mayores valores se corresponden con modelos cuyos clusters se encuentran mejor definidos.
    \begin{equation}
        S = \frac{\sum_{i=1}^{N}{\frac{b_{i}-a_{i}}{\max(a_i,b_i)}}}{N}
    \end{equation}
\end{enumerate}

\subsubsection{Complejidad espacial y temporal}

Los requerimientos de espacio de memoria para K-Means son relativamente pequeños puesto que solamente los puntos de datos y los centroides son almacenados por el algoritmo.
Específicamente, la cantidad de memoria empleada es $O((n+K)m)$, donde $n$ es el número de puntos y $m$ la cantidad de atributos (dimensionalidad) de estos.
Los requisitos de tiempo de este algoritmo son igualmente bajos, es básicamente lineal respecto al tamaño del conjunto de datos.
En particular, el tiempo requerido es $O(I \cdot K \cdot m \cdot n)$, donde $I$ es el número de iteraciones necesarias para converger.
A menudo $I$ es suficientemente pequeño, y usualmente puede ser considerado como un valor constante y despreciable.
De esta forma, K-Means es lineal respecto al tamaño del conjunto de datos $n$, y es muy eficiente siempre que el número de clusters $K$ sea significativamente menor que $n$~\cite{Tan05}.

\subsection{Mini-batch K-Means}\label{subsec:miniBatchKMeans}

Mini-batch K-Means es una variante del algoritmo K-Means que emplea \textit{mini-batches} con el fin de reducir el tiempo de computación, sin afectar la función objetivo a optimizar.
Los \textit{mini-batches} son subconjuntos del conjunto de datos sobre el que se aplica el algoritmo, tomados mediante un muestreo aleatorio en cada iteración.
Estos reducen drásticamente la cantidad de cómputo necesario para converger a un óptimo local.

\begin{algorithm}
    \caption{Mini-batch K-Means}
    \label{algorithm:MiniBatchKMeans}
    Seleccionar $K$ puntos como centroides\;
    \Repeat{Se cumple criterio de convergencia}{
    Tomar una muestra aleatoria de $b$ puntos del conjunto de datos\;
    Asignar cada punto de la muestra al cluster que corresponda al centroide más próximo\;
    Recomputar el centroide de cada cluster\;
    }
\end{algorithm}

Al recomputar los centroides durante cada iteración se tienen en cuenta tanto los puntos de la muestra recién asignados como los asignados durante las iteraciones anteriores.

El algoritmo Mini-batch K-Means converge a mayor velocidad que K-Means, aunque la calidad de sus resultados es menor.
No obstante, para aplicaciones prácticas, esta diferencia en calidad suele ser poco significativa~\cite{Sculley10}.

\section{Clustering Jerárquico Aglomerativo}\label{sec:clusteringJerárquicoAglomerativo}
Las técnicas de clustering jerárquico son, como las de clustering particional, relativamente antiguas en comparación con muchas otras que analizaremos más adelante.
Aún así se mantienen entre las más usadas en la actualidad.
Fueron desarrolladas con la finalidad de resolver algunas de las desventajas de los algoritmos de clustering particional que hemos visto, como su necesidad de una cantidad de clusters prefijada y su naturaleza no determinista.

Existen dos clasificaciones para los algoritmos de clustering jerárquico de acuerdo al modo de generar los clusters:

\begin{itemize}
    \item \textbf{Aglomerativos}: Cada punto comienza en un cluster propio y, en cada paso, se une el par de clusters más próximos entre sí.
    Requiere la definición de una medida para la proximidad entre clusters.
    \item \textbf{Divisivos}: Se comienza con un único cluster que contiene a todos los puntos y, en cada paso, se divide un cluster hasta obtener clusters de solamente un elemento.
    Requiere un criterio para decidir cuál cluster dividir en cada paso y cómo redistribuir a sus miembros entre los dos clusters resultantes.
\end{itemize}

Debido al modo en que se ejecutan los algoritmos de clustering jerárquico, su ejecución sobre un conjunto de datos puede ser visualizado gráficamente mediante diagramas llamados \textit{dendrogramas}, que representan las relaciones entre cada cluster y los sub-clusters formados a partir de este durante los distintos pasos del algoritmo.

Las técnicas de clustering aglomerativas son las más estudiadas~\cite{Tan05}, y serán las que abordaremos en esta sección.

\begin{algorithm}
    \caption{Clustering aglomerativo}
    \label{algorithm:clusteringAglomerativo}
    Computar la matriz de proximidad si es necesario\;
    \Repeat{Solo queda un cluster}{
    Unir los dos clusters más próximos\;
    Actualizar la matriz de proximidad en correspondencia con las distancias entre el nuevo cluster y los ya existentes\;
    }
\end{algorithm}

\subsection{Proximidad entre clusters}\label{subsec:proximidadEntreClusters}

Resulta clave para el funcionamiento del algoritmo~\ref{algorithm:clusteringAglomerativo} el cálculo de la proximidad, o distancia, entre dos clusters.
Este concepto constituye la diferencia entre los distintos algoritmos de clustering aglomerativo, siendo los dos siguientes los métodos más populares~\cite{Aggarawal13}:

\begin{itemize}
    \item \textbf{Single link}: La distancia, o similaridad, entre dos clusters es la distancia entre sus miembros más cercanos entre sí.
    Este método prioriza las regiones en que los clusters son más próximos, sin tener en cuenta la estructura global de estos.
    \item \textbf{Complete link}: La similaridad entre dos clusters se mide por la distancia entre sus miembros más alejados entre sí.
    Esto es equivalente a seleccionar el par de clusters que minimiza el diámetro del resultado de su unión, produciendo clusters compactos generalmente.
\end{itemize}

\begin{figure}[!h]
    \centering
    \includegraphics[width=\textwidth]{agglomerative-clustering.png}
    \caption{Ejemplo de aplicación de algoritmos de clustering aglomerativo. (a) Matriz de similaridad para un conjunto de datos de 4 puntos.
    Dendrogramas obtenidos a partir de este conjunto usando los métodos de clustering aglomerativo (b) single link y (c) complete link. (Tomado de~\cite{Aggarawal13}.)}
\end{figure}

\textbf{Group averaged}~\footnote{También conocido como \textit{Average linkage}} es una tercera variante de clustering aglomerativo.
En este caso, la similitud entre dos clusters no es determinada por la distancia entre dos puntos específicos de estos, sino por el promedio de todas las distancias punto a punto entre los dos clusters.
Se caracteriza por requerir una complejidad de cálculos mayor que las dos variantes anteriormente mencionadas, lo que hace que sea menos usada.
Una variación que busca compensar esta dificultad es la técnica nombrada \textbf{Centroid-based agglomerative clustering}~\footnote{\textit{Clustering aglomerativo basado en centroides} en español}, que determina la distancia entre dos clusters a partir de sus respectivos centroides.

\subsubsection{Método de Ward}

El método de Ward es una propuesta para el cálculo de la distancia entre dos clusters usado en el clustering aglomerativo.
Utiliza la ecuación~\ref{eq:SSE} de la SSE para medir la distancia entre dos clusters como el incremento que produce en este valor unir dichos clusters.
Al igual que K-Means, este método intenta minimizar la suma de las distancias al cuadrado de cada punto al centroide de su cluster.

\subsubsection{Complejidad espacial y temporal}

La matriz de proximidad usada por los algoritmos de clustering jerárquico aglomerativos requiere el almacenamiento de $\frac{1}{2}n^2$ valores (asumiendo que la matriz es simétrica), donde $n$ es la cantidad de observaciones presentes en los datos.
El espacio necesario para almacenar el registro de los clusters es proporcional al número de estos, que es menor o igual que $n$.
La complejidad de memoria requerida por estos algoritmos es, por tanto, $O(n^2)$.

En cuanto al tiempo, el cálculo de la matriz de distancias es $O(n^2)$.
Los pasos 3 y 4 requieren $n-1$ iteraciones puesto que existen $n$ clusters al comienzo, y en cada paso se unen dos, reduciéndose en uno la cantidad de clusters en cada momento.
Si el paso 3 es realizado como una búsqueda lineal sobre la matriz, entonces este requerirá un tiempo $O((n-i+1)^2)$ en la $i$-ésima iteración.
El paso 4 solo requiere un tiempo $O(n-i+1)$ para actualizar la matriz luego de unir dos clusters.
Sin efectuar ninguna modificación, esto nos lleva a una complejidad temporal total $O(n^3)$.
Aunque aplicando mejoras como el empleo de estructuras de datos alternativas para almacenar la matriz y optimizar la búsqueda del paso 3, esta complejidad puede disminuirse hasta $O(n^2\log{n})$~\cite{Tan05}.

La complejidad espacial y temporal de los algoritmos de clustering jerárquico constituye una de las principales limitantes para su empleo sobre conjuntos de datos de tamaño significativo.

\section{Clustering Basado en Densidad}\label{sec:Dbscan}
Algoritmos como K-Means presentan dificultades para identificar clusters cuando las distancias entre elementos de un mismo conjunto es mayor que la de elementos de conjuntos distintos.
En estos casos, puesto que K-Means busca minimizar la distancia entre los elementos dentro de un mismo cluster, producirá clusters que difieran significativamente de los grupos <<correctos>>.
Podemos observar un ejemplo en el escenario de la figura~\ref{img:kmeans-dbscan}, donde se compara el resultado de K-Means con el del algoritmo que analizamos en esta sección, conocido como \textit{DBSCAN}.

\begin{figure}[!h]
    \centering
    \includegraphics[width=\textwidth]{kmeans-dbscan.png}
    \caption{Resultados de los algoritmos K-Means y DBSCAN ejecutados sobre un conjunto de datos que sigue una distribución anisotrópica.}
    \label{img:kmeans-dbscan}
\end{figure}

En la imagen podemos observar asimismo el comportamiento de otro algoritmo sobre el mismo conjunto de datos.
En esta sección abordamos el algoritmo en cuestión, denominado \textit{DBSCAN}~\footnote{Siglas en inglés de \textit{Density-based spatial clustering of applications with noise}.}, que forma parte del conjunto de algoritmos de clustering basados en la densidad de los datos.

Un cluster basado en el criterio de densidad de los puntos consiste en un área densa de puntos conectados, separado de otros clusters por áreas de menor densidad.

\subsection{Densidad}\label{subsec:densidad}

El algoritmo DBSCAN define la densidad alrededor de un punto como la cantidad de puntos localizados alrededor de este en un radio, $Eps$, específico.
El propio punto es incluido en este conteo.
En la figura~\ref{img:dbscan} se puede observar gráficamente esta definición.
En este caso número de puntos alrededor de $A$ es 7.

El valor del radio es determinante en la densidad de un punto.
Si este valor es suficientemente grande, entonces todos los puntos tendrán una densidad de $n$, el número de puntos en el conjunto de datos.
En cambio, si el radio es demasiado pequeño, la densidad de todos los puntos será igual a 1.
Más adelante discutiremos algunas estrategias para la selección de valores apropiados para el radio. % TODO Check if accomplished

\begin{figure}[!h]
    \centering
    \includegraphics[width=\textwidth]{dbscan.png}
    \caption{Densidad en el entorno de un punto y clasificaciones de los puntos según su densidad.}
    \label{img:dbscan}
\end{figure}

De acuerdo con la densidad de un punto, estos pueden ser clasificados de la siguiente forma:

\begin{itemize}
    \item \textbf{Puntos núcleo}: Constituyen puntos de la región interna de un cluster basado en densidad. Un punto es núcleo si el número de puntos alrededor de este (incluyéndolo) supera o iguala un valor $MinPts$, especificado por el usuario. En la figura~\ref{img:dbscan} los puntos identificados con la letra $A$ son núcleos para el radio $Eps$ indicado si $MinPts\leq 7$.
    \item \textbf{Puntos frontera}: Un punto frontera es aquel que no cumple el criterio de núcleo, pero que forma parte de la vecindad de al menos uno de estos. En la figura~\ref{img:dbscan} $B$ es un punto frontera.
    \item \textbf{Puntos de ruido}: Un punto es de ruido si no es núcleo o frontera. En la figura~\ref{img:dbscan} $C$ es un punto de ruido.
\end{itemize}

\subsection{Algoritmo DBSCAN}

A partir de las definiciones dadas de puntos núcleos, fronteras y de ruido, podemos describir el algoritmo DBSCAN del siguiente modo: Todo par de puntos núcleos cuya distancia sea no mayor que $Eps$ son asignados al mismo cluster. De igual forma, los puntos fronteras son asignados al cluster de los puntos núcleos cuya distancia a estos sea menor o igual que $Eps$. (En caso de estar en la vecindad de núcleos pertenecientes a clusters diferentes, un criterio específico debe ser determinado al programar el algoritmo). Los puntos de ruido son descartados y no asignados a ningún cluster.

\section{Clustering Basado en Probabilidades}\label{sec:clusteringBasadoEnProbabilidades}
Los algoritmos de clustering probabilísticos modelan el conjunto de datos a partir de la asunción de que estos son generados a partir de la combinación de determinadas distribuciones de probabilidad.
Estos algoritmos transforman el problema de clustering en el de estimar los parámetros para $K$ distribuciones de probabilidad.
Luego los puntos del conjunto de datos que se correspondan con una misma distribución se asociarán al mismo cluster.

En esta sección analizamos un modelo de clustering probabilístico ampliamente estudiado, conocido como \textit{Gaussian Mixture Model} (GMM) y la técnica \textit{Expectation-maximization}, empleada para estimarlo computacionalmente.

\subsection{Combinación de modelos}\label{subsec:mixtureModels}

Sea $X={x_1,\dots,x_N}$ un conjunto de datos de $N$ observaciones de una variable aleatoria $x$ con $D$ dimensiones.
Asumimos que la variable $x_i$ sigue una distribución consistente con la combinación de $K$ \textit{distribuciones componentes} (clusters), cada una instancia de una distribución para determinados parámetros.
Podemos definir entonces la función de densidad de $x_i$ como:

\begin{equation}
    \label{eq:mixtureModels}
    p(x_i)=\sum_{k=1}^{K}{\pi_k p(x_i|\theta_k)}
\end{equation}

donde cada $\theta_k$ es el conjunto de parámetros específicos de la $k$-ésima componente y $p(x_i|\theta_k)$ su función de densidad.
Los pesos $\pi_k$, también conocidos como \textit{mixing probabilities}, deben satisfacer las condiciones $0\leq \pi_k \leq 1$ y $\sum_{k=1}^{K}{\pi_k}=1$.

Si bien la definición no establece ninguna restricción en cuanto al tipo de distribución que debe seguir cada componente;
en la práctica, para simplificar el estudio de estos modelos, suele asociarse una misma distribución a todas las componentes, variando únicamente sus parámetros.

\subsection{Gaussian Mixture Model}\label{subsec:GMM}

El modelo de clustering probabilístico más extendido es el de combinación de distribuciones normales, conocido en la literatura como \textit{Gaussian Mixture Model} (GMM)~\cite{Murphy12}.
Es asimismo, uno de los modelos de mayor uso en aplicaciones relacionadas con el análisis acústico. %TODO cite

La distribución normal, en el caso de una variable unidimensional $x$, tiene una función de densidad de la forma:

\begin{equation}
    \label{eq:singleGaussian}
    \mathcal{N}(x|\mu,\sigma^2)=\frac{1}{(2\pi\sigma^2)^{1/2}}\exp{(-\frac{1}{2\sigma^2}((x-\mu)^2)}
\end{equation}

donde $\mu$ es la media y $\sigma^2$ la varianza.
Para el caso de $D$ dimensiones, la función toma la forma:

\begin{equation}
    \label{eq:multidimGaussian}
    \mathcal{N}(x|\mu,\Sigma)=\frac{1}{(2\pi)^{D/2}|\Sigma|^{1/2}}\exp{(-\frac{1}{2}(x-\mu)^T \Sigma^{-1}(x-\mu))}
\end{equation}

donde $\mu$ es el vector $D$-dimensional de medias y $\Sigma$, de dimensión $D\times D$, la matriz de covarianza con determinante $|\Sigma|$.

En GMM cada componente corresponde a una distribución normal con determinados valores asociados a sus parámetros $\mu$ y $\Sigma$.
A partir de la ecuación~\ref{eq:mixtureModels} podemos entonces formular este modelo como:

\begin{equation}
    \label{eq:GMM}
    p(x_i|\Theta) = p(x_i|\pi,\mu,\Sigma)= \sum_{k=1}^{K}{\pi_k \mathcal{N}(x_i|\mu_k,\Sigma_k)}
\end{equation}

Para estimar los parámetros de un modelo, podemos emplear el método de máxima verosimilitud.
Dado un conjunto de observaciones $X$, la función de log-verosimilitud se define como:

\begin{equation}
    \label{eq:log-likelihood}
    l(\Theta|X) = \log{p(X|\Theta)} = \sum_{i=1}^{N}{\log{p(x_i|\Theta)}} = \sum_{i=1}^{N}{\log{\sum_{k=1}^{K}{\pi_k \mathcal{N}(x_i|\mu_k,\Sigma_k)}}}
\end{equation}

El método de máxima verosimilitud estima $\Theta$ como el valor que maximiza~\ref{eq:log-likelihood}.
Sin embargo, la optimización de la función~\ref{eq:log-likelihood} presenta serios inconvenientes y generalmente solo pueden ser obtenidos mínimos locales~\cite{Aggarawal13,Murphy12}.

%\subsection{Expectation-maximization}\label{subsec:EM}




\section{Clustering Espectral}\label{sec:clusteringEspectral}
Los algoritmos de \textit{clustering espectral}, a diferencia de otros como K-Means o GMM, no generan necesariamente clusters de estructura convexa, por lo que pueden aplicarse en situaciones de mayor complejidad.

Un algoritmo de clustering espectral se basa en el análisis del espectro (vectores propios) de la matriz laplaciana correspondiente a las similitudes (o distancias) del conjunto de datos.

\begin{algorithm}
    \caption{Clustering Espectral}
    \label{algorithm:SpectralClustering}
    Construir una matriz de similaridad para todos los puntos del conjunto de datos\;
    Convertir el espacio de los puntos a otro, donde se encuentren mejor agrupados, empleando los valores propios de la matriz laplaciana\;
    Emplear un algoritmo de clustering (por ejemplo, K-Means) para particionar la proyección obtenida en el paso anterior\;
\end{algorithm}

A la representación del conjunto de datos obtenida luego del paso 2 se le conoce como \textit{proyección espectral} (\textit{spectral embedding}), y puede asimismo ser empleada con otros fines, como la reducción de dimensiones.

\subsection{Matriz de Similaridad}\label{subsec:matrizDeSimilaridad}

La matriz de similitudes entre los puntos del conjunto de datos suele tomarse como la matriz de adyacencia correspondiente a uno de los siguientes grafos denotados por $G$~\cite{Aggarawal13}:

\begin{enumerate}
    \item \textbf{Grafo de KNN}: Se conectan los puntos $x_i$ y $x_j$ si uno se encuentra entre los $K$ puntos más cercanos al otro.
    La distancia se computa empleando la representación original de los puntos;
    a menudo se utilizan las normas $L_1$ o $L_2$, o la similitud coseno.
    Esto puede hacerse tanto si ambos puntos son $K$-vecinos entre sí, como si uno solo lo es del otro.
    Igualmente el peso de las aristas puede tomarse binario (1 si hay arista, 0 si no) o a partir de la distancia existente entre los puntos.

    \item \textbf{Grafo de $\epsilon$-vecindades}: Dos puntos $x_i$ y $x_j$ se encuentran conectados solo cuando la distancia $|| x_i - x_j ||^2$ es menor que el valor $\epsilon$.

    \item \textbf{Grafo completo}: Todos los puntos con similaridad positiva entre sí son conectados.
    Frecuentemente los pesos de las aristas se definen mediante la función RBF\footnote{\textit{Radial basis function} o \textit{función de base radial} en español.} Gaussiana:
    \[
        W_{ij} = e^{-\sigma \cdot dist(x_i , x_j)^2}
    \]
    donde $dist(x_i , x_j)$ es la distancia euclidiana entre los puntos, y $\sigma$ es un parámetro que determina el decaimiento de la función a medida que los valores se acercan o alejan al 0.
\end{enumerate}

\subsection{Clustering Espectral No Normalizado}\label{subsec:clusteringEspectralNoNormalizado}

Para cada punto $x_i$, su \textit{grado} puede definirse como la suma de los pesos de las aristas incidentes sobre él:

\begin{equation*}
    d_i = \sum_{j=1}^{n}{W_{ij}}
\end{equation*}

A partir de los grados, puede definirse entonces la \textit{matriz de grados} $D$, como la matriz diagonal que satisface $D_{ii}=d_i$.

El \textit{vector indicador} para un subconjunto $A$ de puntos, se denota como $\mathbbm{1}_A = (f_1,\dots,f_n)^T$, donde $f_i = 1$ si $x_i$ pertenece a $A$, y $f_i = 0$ en caso contrario.

La \textit{matriz laplaciana} $L$ se define como:

\begin{equation*}
    L = D - W
\end{equation*}

La matriz laplaciana $L$ cumple las siguientes propiedades~\cite{Luxburg07}:

\begin{enumerate}
    \item Para todo vector $f\in \mathbb{R}^n$, se cumple que
    \[
        f^T Lf = \frac{1}{2}\sum_{i,j=1}^{n}{W_{ij}(f_i - f_j)^2}
    \]
    \item $L$ es simétrica y semidefinida positiva.
    \item El menor valor propio de $L$ es 0, y el vector propio correspondiente es el vector constante uno, $\mathbbm{1}$.
    \item $L$ tiene $n$ valores propios reales no negativos $0=\lambda_1 \leq \lambda_2 \leq \dots \leq \lambda_n$.
\end{enumerate}

Asumiendo que $G$ tiene $K$ componentes conexas $A_1,A_2,\dots,A_K$, en tal caso, sin perder la generalidad, puede representarse $L$ como una matriz diagonal de bloques:

\begin{equation*}
    L =
    \begin{bmatrix}
        L_1 & & &     \\
        & L_2 & &     \\
        & & \ddots &     \\
        & & & L_K
    \end{bmatrix}
\end{equation*}

Dado que $L$ es una matriz diagonal de bloques, su espectro está dado por la unión de los espectros de las matrices $L_i$, de forma que sus vectores propios estarán dados por el conjunto de los de cada una de las matrices $L_i$, con ceros en las posiciones correspondientes a los demás bloques.
Luego por la propiedad 3, se tiene que cada $L_i$ tiene un valor propio 0, que tiene asociado un vector propio con 1 en la i-ésima componente y 0 en las restantes.
Por tanto, la matriz $L$ tiene tantos valores propios 0 como componentes conexas existan en el grafo $G$.


