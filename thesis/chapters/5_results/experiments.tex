\section{Conjuntos de datos}\label{sec:datasets}

Para desarrollar los experimentos, se conformaron 3 conjuntos de datos con segmentos correspondientes a vocalizaciones de especies de murciélagos, aves e insectos.
En las tablas~\ref{table:bats},~\ref{table:birds} y~\ref{table:insects} se describen las especies que componen cada uno de estos conjuntos.
Un cuarto conjunto de datos fue constituido a partir de la unión de los anteriores.
Todos los segmentos se extrajeron de forma manual a partir de grabaciones de mayor duración.

La diversidad de sonidos que pueden producir los individuos de una misma especie, hace que en la clasificación de los segmentos no baste con recoger simplemente el nombre de la especie de la que proceden.
Por tal razón, se empleó además un identificador para distinguir entre tipos de vocalizaciones diferentes aun cuando estas pudieran provenir de individuos pertenecientes a la misma especie.

\section{Resultados}\label{sec:results}

% TODO Do this
Lorem ipsum dolor sit amet, consectetur adipiscing elit, sed eiusmod tempor incidunt ut labore et dolore magna aliqua.
Ut enim ad minim veniam, quis nostrud exercitation ullamco laboris nisi ut aliquid ex ea commodi consequat.
Quis aute iure reprehenderit in voluptate velit esse cillum dolore eu fugiat nulla pariatur.
Excepteur sint obcaecat cupiditat non proident, sunt in culpa qui officia deserunt mollit anim id est laborum.
