%TODO Te falta el objeto de estudio que son los algoritmos de clasificación y los campos de acción es donde lo aplica en la detección de sonido

Desde su origen, el ser humano se ha visto interesado por el medio que le rodea.
Una parte de este medio a la que le ha prestado especial atención son los seres vivos que lo habitan, en especial los animales, con los que guarda mayor parentesco.
Su estudio le ha permitido comprender el modo en que funciona la naturaleza, y puesto que es parte de esta, el modo en que funciona su propio organismo.

La Biología no ha permanecido ajena a lo que ocurre con el resto de las ramas de la ciencia, se ha divido en áreas dedicadas a subcampos específicos, que mutan con el tiempo cual especies vivientes y que, en ocasiones se cruzan con otras ramas para dar origen a nuevos campos de estudio.
El anterior es el caso de la Bioacústica, hija de la Biología y la Acústica, dedicada a la comprensión de señales sonoras de origen biológico.

Durante años la Bioacústica vio su labor dificultada por problemas de índole tecnológica, el desarrollo científico estuvo lastrado por el de los medios necesarios para procesar la información disponible.
Esto ha cambiado recientemente con la introducción de dispositivos de grabación y almacenamiento de mayor calidad y capacidad y, especialmente, con la aplicación de técnicas de Inteligencia Artificial.

Un factor que afectó el avance científico en esta área era que las señales debían ser procesadas por expertos en la materia, quienes tenían la engorrosa tarea de segmentar\footnote{Segmentar es el procedimiento de separar una grabación sonora en porciones (segmentos) en las que ocurren una o varias señales acústicas de interés.} y clasificar cada grabación con el propósito de determinar la ocurrencia de sonidos emitidos por ciertas especies de animales en estas.
Los especialistas debían dedicar horas a este tedioso procedimiento, haciendo poco factible el desarrollo de estudios que requirieran la detección de especies mediante tal proceder.
Dichos estudios son muy necesarios para el desarrollo de proyectos ecológicos, medioambientales y de investigación de nuevas especies, sobre todo en lugares donde las condiciones geográficas y del entorno dificultan la observación directa.

La Inteligencia Artificial ha sido vista como una posible solución a la problemática antes planteada.
Son numerosos los estudios llevados a cabo con diferentes algoritmos de esta para su aplicación en la detección y clasificación de señales bioacústicas~\cite{Gerhard03}.
En general estos pueden ser clasificados en supervisados o no supervisados, de acuerdo a si requieren un conjunto de datos previamente procesados a partir de los cuales <<deducir>> el resultado para nuevos datos o no, respectivamente.
Algoritmos supervisados han sido ampliamente aplicados para la solución del problema en cuestión, quedando la pertinencia y calidad del uso de los no supervisados como un objeto de estudio menos abordado.

Entre las técnicas más estudiadas para la clasificación de señales bioacústicas encontramos~\cite{Gerhard03,Deecke05,Dunkel06,Ilyas14,Lasseck14}:
\begin{itemize}
    \item Neural Networks
    \item K-Nearest Neighbours
    \item Support Vector Machines
    \item Decision Trees
    \item Naive-Bayes
\end{itemize}

Todas las técnicas antes mencionadas pertenecen al campo del aprendizaje supervisado por lo que, para su aplicación, están sujetas a la existencia de un conjunto de entrenamiento previamente etiquetado por expertos.

Son múltiples los problemas que debe enfrentar un sistema automatizado para la clasificación de una señal bioacústica: el ruido de fondo producido por especies emitiendo sonidos de forma simultánea;
el que producen eventos meteorológicos como la lluvia o el viento;
o las variaciones en la frecuencia de la señal que emite un individuo en dos momentos distintos, o diferentes individuos de una misma especie.
Todas constituyen dificultades a tratar cuando se desarrolla un mecanismo para responder a la interrogante de qué especie animal se escucha en un segmento de una grabación dada.

Una tendencia en numerosos trabajos es enfocarse en la clasificación de una categoría taxonómica específica como aves~\cite{Lasseck14,Oliveira15,Stowell14} o primates~\cite{Heinicke15}.
Lo anterior persigue lograr un mejor ajuste en los parámetros de los algoritmos y alcanzar así resultados de mayor calidad.
Se encuentra menos estudiada la aplicación en grabaciones sobre una comunidad de organismos diversa, donde los resultados han sido menos alentadores~\cite{Ilyas14}; y resulta un campo particularmente propicio para la aplicación de algoritmos no supervisados, dada su naturaleza de no requerir conocimiento previo de las categorías esperadas, y que funcionan mejor para mayores separaciones entre las categorías.

Planteado el contexto sobre el que se desarrolla este trabajo, formulamos la siguiente pregunta científica: ¿En qué escenarios resulta propicia la aplicación de algoritmos de aprendizaje no supervisado para la clasificación automática de señales sonoras?

Para dar respuesta a esta interrogante, nos proponemos cumplir los siguientes objetivos:
\begin{itemize}
    \item Presentar un estudio de los algoritmos de aprendizaje no supervisado y clustering semi-supervisado\footnote{Técnica que aplica algoritmos de aprendizaje no supervisado a conjuntos de datos en los que se conoce la categoría de algunos elementos.} y su aplicación en el procesamiento de señales.
    \item Implementar variantes de los algoritmos estudiados.
    \item Establecer una comparación entre los resultados obtenidos, en correspondencia con el problema general (clasificación de señales de diversas especies) y restricciones de este (clasificación de señales en categorías de especies específicas).
\end{itemize}