\section{Validación del clustering}\label{sec:validation}

\begin{frame}
    \frametitle{Validación supervisada}

    \pause
    \begin{block}{Homogeneidad y Completitud}
        Generalizan al clustering las medidas de Teoría de la Información \textit{precisión} y \textit{recobrado}.
    \end{block}

    \pause
    \begin{block}{Información Mutua o \textit{Mutual Information}}
        Mide el nivel en que una variable aleatoria describe otra.
        Puede aplicarse para evaluar un clustering considerando como variables el resultado y las clases verdaderas.
    \end{block}

    \pause
    \begin{block}{Índice de Rand o \textit{Rand Index}}
        Evalúa el cumplimiento del principio de que todo par de objetos incluidos en un mismo cluster, debe pertenecer a la misma categoría y viceversa.
    \end{block}

\end{frame}

\begin{frame}
    \frametitle{Validación no supervisada}
    \framesubtitle{Coeficiente de silueta}

    \pause
    \begin{equation*}
        s_x = \frac{b_x - a_x}{\max{(a_x, b_x)}}
    \end{equation*}

    \pause
    \begin{itemize}
        \item $a_x$: distancia promedio del punto a los puntos del mismo cluster
        \item $b_x$: distancia promedio del punto a los puntos de otros clusters
    \end{itemize}

    \pause
    A diferencia de los medidores anteriores, el coeficiente de silueta oscila entre -1 y 1.

\end{frame}