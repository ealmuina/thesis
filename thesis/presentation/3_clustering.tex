\section{Clustering}\label{sec:clustering}

\begin{frame}
    \frametitle{Clustering}

    \begin{itemize}
        \item<1-> Algoritmos de aprendizaje no supervisado que agrupan los elementos de una colección de datos en conjuntos (\textbf{clusters}).
        \item<2-> Simplificar la estructura del conjunto de datos.
        \item<3-> Encontrar grupos de especial significación.
    \end{itemize}

    \begin{itemize}
        \item<4-> ¿Qué constituye un cluster?
        \item<5-> ¿Cómo hallarlos eficientemente?
    \end{itemize}

\end{frame}

\begin{frame}
    \frametitle{Clustering}

    \begin{columns}
        \column{0.5\textwidth}

        \begin{itemize}
            \item<2-> \alert{Clustering Particional}
            \begin{itemize}
                \item<3-> K-Means
            \end{itemize}

            \item<4-> \alert{Clustering Jerárquico}
            \begin{itemize}
                \item<5-> Aglomerativos
                \item<5-> Divisivos
            \end{itemize}

            \item<6-> \alert{Clustering Basado en Densidad}
            \begin{itemize}
                \item<7-> DBSCAN
                \item<7-> HDBSCAN
            \end{itemize}
        \end{itemize}

        \column{0.5\textwidth}

        \begin{itemize}
            \item<8-> \alert{Clustering Basado en Probabilidades}
            \begin{itemize}
                \item<9-> Gaussian Mixture Model
            \end{itemize}

            \item<10-> \alert{Clustering Basado en Grafos}
            \begin{itemize}
                \item<11-> Clustering Espectral o \textit{Spectral Clustering}
                \item<11-> Affinity Propagation
            \end{itemize}
        \end{itemize}

    \end{columns}


\end{frame}