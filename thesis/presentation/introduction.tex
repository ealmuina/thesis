\begin{frame}
    \frametitle{Introducción}

    \begin{block}{Bioacústica}
        Ciencia multidisciplinaria que combina la biología y la acústica.
        Usualmente se refiere a la investigación de la producción del sonido, su dispersión a través de un medio y su recepción en animales (incluyendo los humanos).
    \end{block}

    \pause

    Enfrenta dificultades como:
    \begin{itemize}
        \item Duración de las grabaciones.
        \item Ruido de fondo.
        \item Variaciones en los parámetros de sonidos de igual clasificación.
    \end{itemize}
\end{frame}

\begin{frame}
    \frametitle{Introducción}

    \begin{columns}
        \column{0.5\textwidth}
        Aprendizaje supervisado:
        \column{0.5\textwidth}
        Aprendizaje no supervisado:
    \end{columns}

    \begin{columns}
        \pause
        \column{0.5\textwidth}

        \begin{itemize}
            \item Algoritmos que requieren una fase previa de \textit{entrenamiento}, utilizando un conjunto de datos clasificados con anterioridad.
            \item Redes neuronales, K-Nearest Neighbours (KNN), Support Vector Machines (SVM), árboles de decisión, Naive Bayes\ldots
        \end{itemize}

        \pause
        \column{0.5\textwidth}

        \begin{itemize}
            \item Algoritmos que no necesitan entrenamiento, por lo que pueden ser empleados cuando no se dispone de un conjunto de datos previamente clasificados.
            \item No etiquetan el resultado con las clases verdaderas puesto que las desconocen.
        \end{itemize}

    \end{columns}
\end{frame}

\begin{frame}
    \frametitle{Introducción}

    Características de los conjuntos de datos:
    \begin{itemize}
        \item Organismos de categorías taxonómicas específicas como \textit{aves} o \textit{primates}.
        \item Comunidades de organismos diversas.
    \end{itemize}

    \begin{block}{Pregunta científica}
        ¿En qué escenarios resulta propicia la aplicación de algoritmos de aprendizaje no supervisado para la clasificación automática de señales sonoras?
    \end{block}
\end{frame}