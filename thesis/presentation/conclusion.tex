\section*{Conclusiones}\label{sec:conclusion}

\begin{frame}
    \frametitle{Conclusiones}

    \begin{itemize}
        \item<2-> Los resultados tienden a ser mejores cuando los MFCC son incluidos.
        \item<3-> La selección del conjunto de características pesa más sobre la calidad del resultado que la del algoritmo.
        \item<4-> Se obtuvo mejor evaluación para el nivel taxonómico intermedio \textit{clase} (aves e insectos).
        \item<5-> Niveles más específico (\textit{orden} de murciélagos) y amplio (aves, insectos y murciélagos) observaron un ligero deterioro en la valoración del resultado.
    \end{itemize}

    \begin{itemize}
        \item<6-> Los resultados obtenidos sugieren que los algoritmos utilizados pueden contribuir al estudio de señales bioacústicas.
    \end{itemize}

\end{frame}

\begin{frame}
    \frametitle{Recomendaciones}

    \begin{itemize}
        \item<2-> Verificar los resultados de este trabajo en conjuntos de datos de diferente composición y mayor tamaño, que permitan dar validez estadística a lo expresado.
        \item<3-> Comprobar la utilidad de la aplicación en pasos intermedios del proceso de clasificación de señales bioacústicas.
        \begin{itemize}
            \item<4-> Selección de los segmentos.
            \item<5-> Mejora del etiquetado de conjuntos de datos.
        \end{itemize}
        \item<6-> Trabajar en la mejora de la interfaz de usuario, de modo que se incrementen las funcionalidades a las que tienen acceso los usuarios y se simplifique su interacción con la misma.
    \end{itemize}

\end{frame}